
\documentclass[
 size=14pt,
 paper=smartboard,  %a4paper, smartboard, screen
 mode=present, 		%present, handout, print
 display=slides, 	% slidesnotes, notes, slides
 style=tuliplab,  	% TULIP Lab style
 pauseslide,
 fleqn,leqno]{powerdot}


\usepackage{cancel}
\usepackage{caption}
\usepackage{stackengine}
\usepackage{smartdiagram}
\usepackage{attrib}
\usepackage{amssymb}
\usepackage{amsmath} 
\usepackage{amsthm} 
\usepackage{mathtools}
\usepackage{rotating}
\usepackage{graphicx}
\usepackage{boxedminipage}
\usepackage{rotate}
\usepackage{calc}
\usepackage[absolute]{textpos}
\usepackage{psfrag,overpic}
\usepackage{fouriernc}
\usepackage{pstricks,pst-3d,pst-grad,pstricks-add,pst-text,pst-node,pst-tree}
\usepackage{moreverb,epsfig,subfigure}
\usepackage{color}
\usepackage{booktabs}
\usepackage{etex}
\usepackage{breqn}
\usepackage{multirow}
\usepackage{natbib}
\usepackage{bibentry}
\usepackage{gitinfo2}
\usepackage{siunitx}
\usepackage{nicefrac}
%\usepackage{geometry}
%\geometry{verbose,letterpaper}
\usepackage{media9}
\usepackage{animate}
%\usepackage{movie15}
\usepackage{auto-pst-pdf}

\usepackage{breakurl}
\usepackage{fontawesome}
\usepackage{xcolor}
\usepackage{multicol}



\usepackage{verbatim}
\usepackage[utf8]{inputenc}
\usepackage{dtk-logos}
\usepackage{tikz}
\usepackage{adigraph}
%\usepackage{tkz-graph}
\usepackage{hyperref}
%\usepackage{ulem}
\usepackage{pgfplots}
\usepackage{verbatim}
\usepackage{fontawesome}


\usepackage{todonotes}
% \usepackage{pst-rel-points}
\usepackage{animate}
\usepackage{fontawesome}

\usepackage{listings}
\lstset{frameround=fttt,
frame=trBL,
stringstyle=\ttfamily,
backgroundcolor=\color{yellow!20},
basicstyle=\footnotesize\ttfamily}
\lstnewenvironment{code}{
\lstset{frame=single,escapeinside=`',
backgroundcolor=\color{yellow!20},
basicstyle=\footnotesize\ttfamily}
}{}


\usepackage{hyperref}
\hypersetup{ % TODO: PDF meta Data
  pdftitle={Presentation Title},
  pdfauthor={Cong Ma},
  pdfpagemode={FullScreen},
  pdfborder={0 0 0}
}


% \usepackage{auto-pst-pdf}
% package to show source code

\definecolor{LightGray}{rgb}{0.9,0.9,0.9}
\newlength{\pixel}\setlength\pixel{0.000714285714\slidewidth}
\setlength{\TPHorizModule}{\slidewidth}
\setlength{\TPVertModule}{\slideheight}
\newcommand\highlight[1]{\fbox{#1}}
\newcommand\icite[1]{{\footnotesize [#1]}}

\newcommand\twotonebox[2]{\fcolorbox{pdcolor2}{pdcolor2}
{#1\vphantom{#2}}\fcolorbox{pdcolor2}{white}{#2\vphantom{#1}}}
\newcommand\twotoneboxo[2]{\fcolorbox{pdcolor2}{pdcolor2}
{#1}\fcolorbox{pdcolor2}{white}{#2}}
\newcommand\vpspace[1]{\vphantom{\vspace{#1}}}
\newcommand\hpspace[1]{\hphantom{\hspace{#1}}}
\newcommand\COMMENT[1]{}

\newcommand\placepos[3]{\hbox to\z@{\kern#1
        \raisebox{-#2}[\z@][\z@]{#3}\hss}\ignorespaces}

\renewcommand{\baselinestretch}{1.2}


\newcommand{\draftnote}[3]{
	\todo[author=#2,color=#1!30,size=\footnotesize]{\textsf{#3}}	}
% TODO: add yourself here:
%
\newcommand{\gangli}[1]{\draftnote{blue}{GLi:}{#1}}
\newcommand{\shaoni}[1]{\draftnote{green}{sn:}{#1}}
\newcommand{\gliMarker}
	{\todo[author=GLi,size=\tiny,inline,color=blue!40]
	{Gang Li has worked up to here.}}
\newcommand{\snMarker}
	{\todo[author=Sn,size=\tiny,inline,color=green!40]
	{Shaoni has worked up to here.}}

%%%%%%%%%%%%%%%%%%%%%%%%%%%%%%%%%%%%%%%%%%%%%%%%%%%%%%%%%%%%%%%%%%%%%%%%
% title
% TODO: Customize to your Own Title, Name, Address
%
\title{Learning Report}
\author{
Cong Ma
\\
\\
\\QingDao Technological University
\\
\\

}
\date{\gitCommitterDate}


% Customize the setting of slides
\pdsetup{
% TODO: Customize the left footer, and right footer
rf=\href{http://www.tulip.org.au}{
Last Changed by: \textsc{\gitCommitterName}\ \gitVtagn-\gitAbbrevHash\ (\gitAuthorDate)
},
cf={Learning Report},
}


\begin{document}

\maketitle




%%==========================================================================================
%%
\begin{slide}[toc=,bm=]{Overview}
\tableofcontents[content=currentsection,type=1]
\end{slide}
%%
%%==========================================================================================


\section{Term summary}


%%==========================================================================================
%%
\begin{slide}{Term summary}
  \begin{itemize}
    \item
    First of all, in the past six months, I have completed the learning tasks of flip 00 and flip 01.
    
    \begin{itemize}
    \item
    I learned to use a lot of software on flip 00.Such as:Latex,Github and Smartgit.
  
    \item
    In flip 01,I have learned some NLP methods and knowledge.
    \end{itemize}

    \item
    In addition,I also completed many graduate foundation courses.

  \end{itemize}

%%==========================================================================================

%%==========================================================================================

\end{slide}

\begin{slide}[toc=,bm=]{Term summary}
  \begin{itemize}
    \item
    Master some basic syntax of Python.
    \item
    Learn and understand the artificial neural network and machine learning algorithm.
    \item
    I think my biggest harvest  is  understanding and realizing some simple algorithms of artificial intelligence.

  \end{itemize}

%%==========================================================================================

%%==========================================================================================

\end{slide}

\section{Future Plans}


%%==========================================================================================
%%
\begin{slide}{Future Plans}
%Related Work - Outlying Aspects Mining
% \begin{itemize}
% \item
% Existing Methods - \textcolor{orange}{Feature selection}

% \begin{itemize}
% \item
% To distinguish two classes:
% the query point (positive) \& rest of data (negative)
% \end{itemize}
% \vspace{1cm}
% \twocolumn[
% \savevalue{lfrheight}=5cm,
% \savevalue{lfrprop}={
% linestyle=solid,framearc=.2,linewidth=1pt},
% rfrheight=\usevalue{lfrheight},
% rfrprop=\usevalue{lfrprop}
% ]{
% Disadvantages
% \begin{itemize}
% \item
% \smallskip
% Positive and negative classes are \textcolor{orange}{Not} balanced.

% \item
% \smallskip
% \textcolor{orange}{Not} quantify the outlying degree accurately.

% \item
% \smallskip
% \textcolor{orange}{Not} identify group outlying aspects.
% \end{itemize}
% }
% {
% Advantages
% \begin{itemize}
% \item
% \smallskip
% Easy to operate.

% \item
% \smallskip
% Resolve dimensionality bias.
% \end{itemize}
% }
% \end{itemize}
\begin{itemize}
  \item
  Learn more machine learning algorithms.
  \item
  Continue to follow the arrangement and complete the study of flip 02 and flip 03.

\end{itemize}

\end{slide}


\section{Thinking and interest in research}

\begin{slide}{Thinking and interest in research}
  \begin{itemize}
    \item
    In my submission,machine learning can be divided into the following steps.
    
    \begin{itemize}
    \item
    Problem definition.
    \item
    Data collection.
    \item
    Data visualization.
    \item
    Data cleaning.
    \item
    Model establishment and optimization.
    \item
    summary.
    \end{itemize}
  \end{itemize}
\end{slide} 


\begin{wideslide}[]{Contact Information}
\centering
\vspace{\stretch{1}}
\twocolumn[
lcolwidth=0.35\linewidth,
rcolwidth=0.65\linewidth
]
{
% \centerline{\includegraphics[scale=.2]{tulip-logo.eps}}
}
{
\vspace{\stretch{1}}
Made By Cong Ma\\
QingDao Technological University\\

% \begin{description}
%  \item[\textcolor{orange}{\faEnvelope}] \href{mailto:gangli@tulip.org.au}
%  {\textsc{\footnotesize{gangli@tulip.org.au}}}

%  \item[\textcolor{orange}{\faHome}] \href{http://www.tulip.org.au}
%  {\textsc{\footnotesize{Team for Universal Learning and Intelligent Processing}}}
% \end{description}
}
\vspace{\stretch{1}}
\end{wideslide}

\end{document}

\endinput
